\documentclass[dvipsnames, svgnames,a4paper,11pt]{article}
\usepackage{todonotes}
\input{Settings} % 导入模板的相关设置
\usepackage{lipsum}
\usepackage{amsmath}
\usepackage{array}
\usepackage{multicol}
\usepackage{tabularx}
\usepackage{graphicx}


\begin{document}

\scoresTable{}{}{}{}{}{}{}{}

% \infoTable{专业}{年级}{姓名}{学号}{实验时间}{教师签名}
\infoTable{物理学}{2022级}{丁侯凯、周新鹏}{22344009、22344191}{2024.9.27}{}

\begin{center}
	\LARGE 实验一 \quad 蓝牙音箱的焊接和调试
\end{center}

\textbf{【实验报告注意事项】}
\begin{enumerate}
	\item 实验报告由三部分组成:
	\begin{enumerate}
		\item 预习报告:(提前一周)认真研读\underline{\textbf{实验讲义}},弄清实验原理;实验所需的仪器设备、用具及其使用(强烈建议到实验室预习),完成课前预习思考题;了解实验需要测量的物理量,并根据要求提前准备实验记录表格(第一循环实验已由教师提供模板,可以打印)。预习成绩低于10分(共20分)者不能做实验。
	    \item 实验记录:认真、客观记录实验条件、实验过程中的现象以及数据。实验记录请用珠笔或者钢笔书写并签名(\textcolor{red}{\textbf{用铅笔记录的被认为无效}})。\textcolor{red}{\textbf{保持原始记录,包括写错删除部分,如因误记需要修改记录,必须按规范修改。}}(不得输入电脑打印,但可扫描手记后打印扫描件);离开前请实验教师检查记录并签名。
	    \item 分析讨论:处理实验原始数据(学习仪器使用类型的实验除外),对数据的可靠性和合理性进行分析;按规范呈现数据和结果(图、表),包括数据、图表按顺序编号及其引用;分析物理现象(含回答实验思考题,写出问题思考过程,必要时按规范引用数据);最后得出结论。
	\end{enumerate}
	\textbf{实验报告就是将预习报告、实验记录、和数据处理与分析合起来,加上本页封面。}
	\item 每次完成实验后的一周内交\textbf{实验报告}(特殊情况不能超过两周)。
	\item 除实验记录外,实验报告其他部分建议双面打印。
\end{enumerate}


\clearpage
\tableofcontents
\clearpage

\setcounter{section}{0}
\nsection{ETII 一}{蓝牙音箱的焊接和调试}{预习报告}
	
\subsection{实验目的}
\begin{enumerate}
	\item 完成蓝牙音箱的焊接和调试。
	\item 基于LM4863放大电路探究放大倍数与输入电压的关系。
	\item 基于LM4863放大电路探究放大倍数与频率的变化关系曲线。
	\item 如何实现功放调音(提升部分,选做)。
\end{enumerate}

\subsection{仪器用具}
\begin{table}[htbp]
	\centering
	\renewcommand\arraystretch{1.6}
	% \setlength{\tabcolsep}{10mm}
	\begin{tabular}{p{0.05\textwidth}|p{0.20\textwidth}|p{0.05\textwidth}|p{0.5\textwidth}}
	\hline
	编号& 仪器用具名称 & 数量 &  主要参数(型号,测量范围,测量精度等) \\
	\hline
	1&电烙铁等焊接仪器 &1 & 实验室提供\\
	2&蓝牙音箱等待焊接设备 &1 & 实验室提供\\
	3&模拟电子相关仪器(如信号发生器、示波器、电脑等) &1 & 实验室提供 \\
	\hline
\end{tabular}
\end{table}

\subsection{原理概述}

		运算放大器的主要功能是对输入信号进行放大、滤波、积分、微分等信号处理。最常见的运算放大电路包括反相放大电路和同相放大电路。
	
		\subsection{反相比例放大电路}
			1)电路原理:

			在反相放大电路中,输入信号施加在运算放大器的反相输入端(-),同相输入端(+)通常接地。输出信号与输入信号的相位相反,故称为反相放大。

			\vspace{1cm}
			电路结构:

			输入电阻 \( R_{\text{in}} \) 连接输入信号 \( V_{\text{in}} \) 和运算放大器的反相端。
			
			反馈电阻 \( R_f \) 连接运算放大器的输出端和反相输入端。
			
			同相端接地。
			\vspace{1cm}
			电路图:
			\begin{figure}[htbp]
				\centering
				\includegraphics[width=0.50\textwidth]{反向放大电路原理图.png}
				\caption{反向放大电路原理图}
				\label{fig:反向放大电路原理图}
			\end{figure}
			
			2)工作原理:

			根据虚短和虚断原理:
			\begin{enumerate}
				\item 虚短:由于理想运算放大器的开环增益 \( A_{\text{open}} \) 无限大,输入端电压差 \( V_- - V_+ \approx 0 \),即 \( V_- = V_+ \)。
				\item 虚断:理想运算放大器的输入端电流为零,即 \( I_- = 0 \)。
				\item 由于同相端接地,故 \( V_+ = 0 \),则 \( V_- \approx 0 \)。根据基尔霍夫电流定律(KCL),通过 \( R_{\text{in}} \) 的电流等于通过反馈电阻 \( R_f \) 的电流:

				\begin{equation}
					\frac{V_{\text{in}} - V_-}{R_{\text{in}}} = \frac{V_- - V_{\text{out}}}{R_f}				
					\label{eq:1}
				\end{equation}

			\end{enumerate}

			3)放大倍数推导:
			
			因为 \( V_- \approx 0 \),所以方程变为:
			
			\begin{equation}
				\frac{V_{\text{in}}}{R_{\text{in}}} = \frac{-V_{\text{out}}}{R_f}			
				\label{eq:2}
			\end{equation}

			解得输出电压 \( V_{\text{out}} \):
			\begin{equation}
				V_{\text{out}} = - \frac{R_f}{R_{\text{in}}} V_{\text{in}}		
				\label{eq:3}
			\end{equation}

			反相放大器的电压增益为:
			\begin{equation}
				A_v = \frac{V_{\text{out}}}{V_{\text{in}}} = - \frac{R_f}{R_{\text{in}}}	
				\label{eq:4}
			\end{equation}
			
			因此,输出电压与输入电压成反比,且增益由反馈电阻和输入电阻的比值决定。

	\subsubsection{同相比例放大电路}
		1)电路原理:
		在同相放大电路中,输入信号施加在运算放大器的同相输入端(+),输出信号与输入信号同相。
		
		反馈电阻 \( R_f \) 连接输出端和反相输入端。
		输入电阻 \( R_{\text{in}} \) 连接反相输入端和接地。
		输入信号施加在同相输入端。

		电路图:

	\begin{figure}[htbp]
		\centering
		\includegraphics[width=0.50\textwidth]{同相放大电路原理图.png}
		\caption{同相放大电路原理图}
		\label{fig:同相放大电路原理图}
	\end{figure}

		2)工作原理:

		同样应用虚短和虚断原理:虚短:由于 \( V_+ = V_- \),故运算放大器两端电压相等。虚断:输入端电流为零。

		输出电压通过反馈调整,使得输入端的电压与输入信号相同。根据电压分压原理,反馈电阻和输入电阻的电流关系满足:

		\begin{equation}
			V_{\text{out}} = \left( 1 + \frac{R_f}{R_{\text{in}}} \right) V_{\text{in}}
			\label{eq:5}
		\end{equation}

		3)放大倍数推导:

		同相放大器的电压增益为:
		\begin{equation}
			A_v = 1 + \frac{R_f}{R_{\text{in}}}
			\label{eq:6}
		\end{equation}

因此,输出电压与输入电压同相,且增益为 \( 1 + \frac{R_f}{R_{\text{in}}} \)。

\clearpage
	\subsubsection{蓝牙音箱主要电路原理}
	\begin{figure}[htbp]
		\centering
		\includegraphics[width=0.90\textwidth]{原理图.png}
		\caption{蓝牙音箱主要电路原理}
		\label{fig:蓝牙音箱主要电路原理}
	\end{figure}




	\subsubsection{参考文献}
	\textcolor{red}{参考《电路基础 (第五版)》,(美)亚历山大著。}


\subsection{实验安全注意事项}
\begin{enumerate}
	\item 注意电烙铁的使用规范,防止烫伤。
	\item 操作电路前需断电。
\end{enumerate}
\clearpage

		%插入图片的事例
	% \begin{figure}[htbp]
	% 	\centering
	% 	\includegraphics[width=0.75\textwidth]{XX图.jpg}
	% 	\caption{XX图}
	% 	\label{fig:XX}
	% \end{figure}
		%插入公式事例
	% \begin{equation}
	% v(t) = s(t) +n(t)
	% \label{eq:1}
	% \end{equation}
	
		%插入多行对齐公式事例
	% \begin{align}
	% 	u_{px}(t) &= u_r(t)u_a(t) \nonumber \\
	% 			  &= A_I[s(t)\sin(\omega_m t + \theta)\sin(\omega_r t) + n(t)\sin(\omega_r t)] \nonumber \\
	% 			  &= A_I { \frac{s(t)}{2} \left[ \cos((\omega_m - \omega_r)t + \theta) - \cos((\omega_m + \omega_r)t + \theta) \right] + n(t)\sin(\omega_r t) } \label{eq:2}
	% \end{align}

%  给出段落间的空行  \vspace{1cm}

	%插入表格事例
% \begin{table}[ht]
	% \centering
	% \begin{tabularx}{\textwidth}{|c|X|X|X|}
	% 	\hline
	% 	% 第一行表头
	% 	时间常数 & R波形  & X波形  & Y波形 \\
	% 	\hline
	% 	% 数据行 1
	% 	10us & \includegraphics[width=3cm]{示波器图片1.1.png} & \includegraphics[width=3cm]{1.2.png}&\includegraphics[width=3cm]{1.3.png} \\
	% 	\hline
	% 	% 数据行 2
	% 	100us &\includegraphics[width=3cm]{2.1.png} & \includegraphics[width=3cm]{2.2.png} & \includegraphics[width=3cm]{2.3.png}\\
	% 	\hline
	% 	% 数据行 3
	% 	1ms &\includegraphics[width=3cm]{3.1.png} & \includegraphics[width=3cm]{3.2.png} & \includegraphics[width=3cm]{3.3.png} \\
	%   \hline
	% \end{tabularx}
	% \caption{陡降为6dB/oct时,改变时间常数,低通滤波器配合乘法器解调过程探究记录}
	% \label{tab:1}
% \end{table}




\clearpage
\begin{table}
	\renewcommand\arraystretch{1.7}
	\centering
	\begin{tabularx}{\textwidth}{|X|X|X|X|}
	\hline
	专业:& 物理学 &年级:& 2019级 \\
	\hline
	姓名: & 丁侯凯、周新鹏& 学号:&22344009、22344191\\
	\hline
	室温:& 24℃& 实验地点: & A522\\
	\hline
	学生签名:& \includegraphics[width=2cm]{签名.jpg}   \includegraphics[width=2cm]{签名1.jpg}   & 评分: &\\
	\hline
	实验时间:& 2024.9.27& 教师签名:&\\
	\hline
	\end{tabularx}
\end{table}

\nsection{ETII 一}{蓝牙音箱的焊接和调试}{实验记录}
\subsection{实验内容和步骤}
	\subsubsection{电路分析}
		\begin{enumerate}
		\item 按照蓝牙音箱说明书进行焊接,并通过连接音箱并播放音乐来测试是否能正常使用,实验记录\textcolor{red}{如图\ref{fig:蓝牙音箱成品图}:}
		\item 测试蓝牙音箱的放大倍数。
		\item 控制变量,探究放大倍数与频率的关系曲线。
		\item 控制变量,在频率f恒定在1kHz时改变电压大小,观察放大倍数,探究放大倍数与输入交流电压的关系。
		\item 观察并记录实验数据。
		\end{enumerate}


	\subsubsection{对LM4863芯片电路分析}
		查找有关资料后,对LM4863芯片内部电路进行电路分析\textcolor{red}{如图\ref{fig:LM4863芯片电路图}:}
	\begin{figure}[htbp]
		\centering
		\includegraphics[width=0.80\textwidth]{LM4863芯片电路.png}
		\caption{LM4863芯片电路图}
		\label{fig:LM4863芯片电路图}
	\end{figure}
	其中,实验时R1等于4.7KΩ。

	\begin{figure}[htbp]
		\centering
		\includegraphics[width=0.65\textwidth]{蓝牙音箱.jpg}
		\caption{蓝牙音箱}
		\label{fig:蓝牙音箱}
	\end{figure}

	\begin{figure}[htbp]
		\centering
		\includegraphics[width=0.65\textwidth]{蓝牙音箱成品图.jpg}
		\caption{蓝牙音箱成品图}
		\label{fig:蓝牙音箱成品图}
	\end{figure}
	
	
	根据反相比例放大电路可知,$Amp1A$的放大倍数$A=-R_f/R$,所以放大倍数$A=-20/4.7≈-4.26$,根据虚短虚断原理,如图标注的位置电压绝对值相同,但由于$Amp2A$是同相比例放大器,所以放大4.26倍,$U_0/U_{in}=4.26-(-4.26)=8.52$倍
		
\clearpage
\begin{table}
	\renewcommand\arraystretch{1.7}
	\begin{tabularx}{\textwidth}{|X|X|X|X|}
	\hline
	专业:& 物理学 &年级:& 2022级\\
	\hline
	姓名: &丁侯凯、周新鹏 & 学号:&22344009、22344191 \\
	\hline
    日期:&2024.9.27 & 评分: &\\
	\hline
	\end{tabularx}
\end{table}

\nsection{ETII 一}{蓝牙音箱的焊接和调试}{分析与讨论}
\subsection{实验数据分析}
	\subsubsection{使用KICAD进行画图}
	利用KICAD对实验电路原理图进行仿真,\textcolor{red}{如图\ref{fig:KICAD进行画图}:}

	\begin{figure}[htbp]
		\centering
		\includegraphics[width=0.75\textwidth]{KICAD进行画图.png}
		\caption{KICAD画图仿真}
		\label{fig:KICAD进行画图}
	\end{figure}

	\subsubsection{探究放大倍数与输入交流电压的关系(在1kHz条件下)}
	\begin{table}[ht]
		\centering
		\begin{tabularx}{\textwidth}{|c|X|X|X|}
			\hline
			% 第一行表头
			交流电压有效值$V_{rms}/mV$ & 偏移电压$V_{DC}/mV$  & 输出电压$U_0/V$  & 放大倍数 \\
			\hline
			% 数据行 1
			20 & 10 & 0.159&7.95 \\
			\hline
			% 数据行 2
			30 &15 &0.241 & 8.03\\
			\hline
			% 数据行 3
			40& 20& 0.322&8.05 \\
			\hline
			% 数据行 4
			50&25 &0.403 & 8.06\\
			\hline
			% 数据行 5
			60& 30& 0.485& 8.08\\
			\hline
			% 数据行 6
			70&35 & 0.565&8.07 \\
			\hline
			% 数据行 7
			80& 40&0.647 & 8.09\\
			\hline
			% 数据行 8
			90&45 & 0.729& 8.10\\
			\hline
			% 数据行 9
			100 & 50& 0.811& 8.11\\
			\hline
			110&55&0.892&8.11\\
			\hline
		\end{tabularx}
	\caption{在f=1kHz条件下,放大倍数与输入交流电压的关系}
	\label{tab:在f=1kHz条件下,放大倍数与输入交流电压的关系}
		\end{table}
		其中,在电压为20$V_{rms}/mV$,数据有较大误差,推测是该电压低于蓝牙音箱的规定工作电压,刨去该数据点对剩余数据进行分析,\textcolor{red}{如图}\ref{fig:放大倍数与输入交流电压关系的数据分析}。

		\begin{figure}[htbp]
			\centering
			\includegraphics[width=0.75\textwidth]{放大倍数与输入交流电压的关系(数据分析).png}
			\caption{放大倍数与输入交流电压关系的数据分析}
			\label{fig:放大倍数与输入交流电压关系的数据分析}
		\end{figure}

		发现图像为线性关系,但其斜率较低,在图像中不断趋于0,因此可以认为在误差范围中,放大倍数趋于不变。计算得其算数平均值:
		\begin{equation}
			A=8.065
			\label{eq:放大倍数}
		\end{equation}

		\subsubsection{探究放大倍数与频率的关系曲线(交流信号有效值:100mV,偏移量:50mV)}
		\begin{table}[ht]
			\centering
			\begin{multicols}{2} % 使用multicol环境进行两列布局
				\begin{tabularx}{\columnwidth}{|c|X|}
					\hline
					% 第一列表头
					频率Hz & 放大倍数  \\
					\hline
					% 数据行 1-20
					10 & 2.29  \\ \hline
					15 & 3.30  \\ \hline
					20 & 4.15  \\ \hline
					25 & 5.43  \\ \hline
					30 & 5.88  \\ \hline
					35 & 6.24  \\ \hline
					40 & 6.54  \\ \hline
					45 & 6.79  \\ \hline
					50 & 7.39  \\ \hline
					70 & 7.70  \\ \hline
					90 & 7.80  \\ \hline
					100 & 8.04  \\ \hline
					150 & 8.14  \\ \hline
					200 & 8.18  \\ \hline
					250 & 8.20  \\ \hline
					300 & 8.21  \\ \hline
					350 & 8.21  \\ \hline
					400 & 8.21  \\ \hline
					450 & 8.21  \\ \hline
					500 & 8.21  \\ \hline
				\end{tabularx}
				
				\columnbreak % 在此处换到第二列
		
				\begin{tabularx}{\columnwidth}{|c|X|}
					\hline
					% 第二列表头
					频率Hz & 放大倍数  \\
					\hline
					% 数据行 21-40
					600 & 8.20  \\ \hline
					700 & 8.18  \\ \hline
					800 & 8.16  \\ \hline
					900 & 8.13  \\ \hline
					1000 & 8.11  \\ \hline
					1500 & 7.97  \\ \hline
					2000 & 7.80  \\ \hline
					2500 & 7.60  \\ \hline
					3000 & 7.39  \\ \hline
					4000 & 6.94  \\ \hline
					5000 & 6.47  \\ \hline
					7000 & 5.59  \\ \hline
					9000 & 4.82  \\ \hline
					10000 & 4.49  \\ \hline
					20000 & 2.51  \\ \hline
					30000 & 1.65  \\ \hline
					40000 & 1.18  \\ \hline
					50000 & 0.84  \\ \hline
					100000 & 0.06  \\ \hline
				\end{tabularx}
			\end{multicols}
			\caption{探究放大倍数与频率的关系曲线}
			\label{tab:探究放大倍数与频率的关系曲线}
		\end{table}

		\begin{figure}[htbp]
			\centering
			\includegraphics[width=0.75\textwidth]{探究放大倍数与频率的关系曲线(数据分析).png}
			\caption{探究放大倍数与频率的关系曲线}
			\label{fig:探究放大倍数与频率的关系曲线}
		\end{figure}
		由\textcolor{red}{如表\ref{tab:探究放大倍数与频率的关系曲线}、如图\ref{fig:探究放大倍数与频率的关系曲线}}可得,放大频率随频率由明显的变大而先上升再下降。



	

\clearpage
\subsection{实验结论与实验心得}
通过焊接蓝牙音箱、分析LM4863功率放大电路的性能、测量电压增益来评估电路的运行情况。

\begin{enumerate}
	\item 根据实验数据表明,LM4863芯片电路的电压增益会随着频率发生变化。在工作的频率范围大致在30Hz时与产品使用说明介绍的正常工作频率吻合。
	\item 但在测量的电压增益较小的高频处,电压增益略小于理论值8.5,可能的原因是电路中的电路元件和耦合电容的影响,以及发大器增益带宽限制、实验室电压的稳定性、负反馈电路中元件的精确度等。
\end{enumerate}
		

\clearpage
% ---------------------------------------------------------------------
%   参考文献
%   注:使用参考文献时应按照xelatex->bibtex->xelatex->xelatex顺序进行编译
\phantomsection
\addcontentsline{toc}{section}{参考文献}
\bibliographystyle{unsrt} % 调整参考文献的格式,可供选择的有 plain, unsrt, alpha, abbrv, ieeetr, acm, siam, apalike 等。
\bibliography{myref}

% \clearpage
% \appendix
% \appendixpage
% \addappheadtotoc

% \begin{tbox}{字体设置(中文)}
% \begin{enumerate}
% 	\item 宋体:{\songti 山有扶苏,隰有荷华}
% 	\item 仿宋:{\fangsong 山有扶苏,隰有荷华}
% 	\item 黑体:{\heiti 山有扶苏,隰有荷华}
% 	\item 楷书:{\kaishu 山有扶苏,隰有荷华}
% \end{enumerate}
% \end{tbox}

% \begin{tbox}{Set font(English)}
% \begin{enumerate}
% 	\item roman:\quad{\rmfamily Hello world!}
% 	\item sans-serif:\quad{\sffamily Hello world!}
% 	\item typewriter:\quad{\ttfamily Hello world!}
% \end{enumerate}
% \end{tbox}

% \begin{tbox}{公式}
% 	无编号公式
%     \begin{equation*}
%         J(\theta) = \mathbb{E}_{\pi_\theta}[G_t] = \sum_{s\in\mathcal{S}} d^\pi (s)V^\pi(s)=\sum_{s\in\mathcal{S}} d^\pi(s)\sum_{a\in\mathcal{A}}\pi_\theta(a|s)Q^\pi(s,a)
%     \end{equation*}
% $$ J(\theta) = \mathbb{E}_{\pi_\theta}[G_t] = \sum_{s\in\mathcal{S}} d^\pi (s)V^\pi(s)=\sum_{s\in\mathcal{S}} d^\pi(s)\sum_{a\in\mathcal{A}}\pi_\theta(a|s)Q^\pi(s,a) $$
%     有编号公式
%     \begin{equation}
%         J(\theta) = \mathbb{E}_{\pi_\theta}[G_t] = \sum_{s\in\mathcal{S}} d^\pi (s)V^\pi(s)=\sum_{s\in\mathcal{S}} d^\pi(s)\sum_{a\in\mathcal{A}}\pi_\theta(a|s)Q^\pi(s,a)
%     \end{equation}
%     \begin{equation}
%         J(\theta) = \mathbb{E}_{\pi_\theta}[G_t] = \sum_{s\in\mathcal{S}} d^\pi (s)V^\pi(s)=\sum_{s\in\mathcal{S}} d^\pi(s)\sum_{a\in\mathcal{A}}\pi_\theta(a|s)Q^\pi(s,a)
%     \end{equation}
% 	波尔文积分
%     \[
%     \begin{cases}
%         \vspace{0.2cm}
%         \displaystyle{\int_{0}^{\infty} \frac{\sin(x)}{x}\dd{x} = \frac{\pi}{2}}\\
%         \vspace{0.2cm}
%         \displaystyle{\int_{0}^{\infty} \frac{\sin(x)}{x} \frac{\sin(x/3)}{x/3}\dd{x} = \frac{\pi}{2}} \\
%         \vspace{0.2cm}\cdot\cdot\cdot\\
%         \vspace{0.2cm}
%         \displaystyle{\int_{0}^{\infty} \frac{\sin(x)}{x} \frac{\sin(x/3)}{x/3} \cdot\cdot\cdot \frac{\sin(x/13)}{x/13}\dd{x} = \frac{\pi}{2}}\\
%         \displaystyle{\int_{0}^{\infty} \frac{\sin(x)}{x} \frac{\sin(x/3)}{x/3} \cdot\cdot\cdot \frac{\sin(x/15)}{x/15}\dd{x} = \frac{467807924713440738696537864469}{935615849440640907310521750000}\pi}
%     \end{cases}  
%     \]
% 	多行对齐公式
% 	\begin{align*} 
% 		\hat{H}^{(2)} &= \frac{1}{2}\sum_{\alpha}\sum_{\beta}\int \dd[3]{x}\dd[3]{x'} \hat{\psi}^\dagger_\alpha(\vb{x})\hat{\psi}_\beta^\dagger(\vb{x}')\qty[\sum_{\vb{q}\neq 0} \frac{4\pi e^2}{q^2}\mathrm{e}^{i\vb{q}\cdot(\vb{x}-\vb{x}')}]\hat{\psi}_{\beta}(\vb{x}')\hat{\psi}_\alpha(\vb{x})\\[.2cm]
% 		&=\frac{1}{2V}\sum_{\vb{k}}\sum_{\vb{k}'}\sum_{\vb{q}\neq 0}\sum_{\alpha}\sum_{\beta}\qty(\frac{4\pi e^2}{q^2})\hat{C}_{\vb{k}+\vb{q},\alpha}^\dagger\hat{C}_{\vb{k}'-\vb{q},\beta}^\dagger\hat{C}_{\vb{k}'\beta}\hat{C}_{\vb{k}\alpha}. 
% 	\end{align*}
% \end{tbox}

% \begin{tbox}{引用}
% 	对公式的引用,如\cref{equ:test}
% 	\begin{equation}
%         J(\theta) = \mathbb{E}_{\pi_\theta}[G_t] = \sum_{s\in\mathcal{S}} d^\pi (s)V^\pi(s)=\sum_{s\in\mathcal{S}} d^\pi(s)\sum_{a\in\mathcal{A}}\pi_\theta(a|s)Q^\pi(s,a)
% 		\label{equ:test}
%     \end{equation}
% 	对图像的引用,如\cref{fig:test}
% 	\begin{figure}[H]
% 		\centering
% 		\includegraphics[width=0.3\textwidth]{example.png}
% 		\caption{测试图片}
% 		\label{fig:test}
% 	\end{figure}
% 	对表格的引用,如\cref{tab:test}
% 	\begin{table}[H]
% 		\renewcommand\arraystretch{1.5}
% 		\caption{一个空表格}
% 		\begin{tabularx}{\textwidth}{|p{0.15\textwidth}|X|X|X|X|}
% 		\hline
% 		 &  &  &  &  \\    
% 		\hline
% 		 &  &  & &  \\    
% 		\hline
% 		\end{tabularx}
% 		\label{tab:test}
% 	\end{table}
% \end{tbox}

% \begin{tbox}{表格}
% 	tabular可以自己更改宽度
% 	\begin{table}[H]
% 		\renewcommand\arraystretch{1.7}
% 		\centering
% 		\caption{一个空表格}
% 		\begin{tabular}{|p{0.15\textwidth}|p{0.15\textwidth}|p{0.15\textwidth}|p{0.15\textwidth}|}
% 		\hline
% 		&   &  &  \\
% 		\hline
% 		 &   &  &  \\
% 		\hline    
% 		\end{tabular}
% 	\end{table}
% 	tabularx可以自适应宽度
% 	\begin{table}[H]
% 		\renewcommand\arraystretch{1.7}
% 		\centering
% 		\caption{一个空表格}
% 		\begin{tabularx}{\textwidth}{|p{0.2\textwidth}|X|X|X|X|X|X|}
% 			\hline
% 			& &  &  &  &  &  \\
% 			\hline
% 			 & &  &  &  &  &  \\
% 			\hline
% 		\end{tabularx}
% 	\end{table}
% \end{tbox}
\end{document}

